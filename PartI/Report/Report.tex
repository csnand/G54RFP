 \documentclass{article}
\usepackage{cite}
\usepackage{inputenc}
\usepackage{setspace}
\usepackage[margin=0.75in]{geometry}
\usepackage[style=numeric]{biblatex}
\addbibresource{../bibs/ref.bib}
\usepackage{float}
\usepackage{graphicx}
\graphicspath{ {./images/} }


\onehalfspace
\setlength{\parindent}{0pt}
\setlength{\parskip}{1em}



\begin{document}

\begin{center}
  \LARGE{\textbf{Real-world Functional Programming}} \\
  \Large{Coursework Part I Report} \\
  \normalsize{14274056 Junsong Yang (psyjy3)} \\
  \today
\end{center}


\begin{normalsize}
  \section{Task I.1}

  \begin{figure}[H]

    \begin{minipage}[b]{0.48\linewidth}
      \centering
      \centerline{\includegraphics[width=8.0cm]{Hamming}}
      % \vspace{1.5cm}
      \centerline{ (a) Hamming Function Definition}\medskip
    \end{minipage}
    \hfill
    \begin{minipage}[b]{0.48\linewidth}
      \centering
      \centerline{\includegraphics[width=8.0cm]{cyclic}}
      % \vspace{1.5cm}
      \centerline{ (b) Cyclic Graph}\medskip
    \end{minipage}
    % 
    \caption{TaskI.1}
    \label{fig:taskI.1}
    % 
  \end{figure}


  The hamming function can be defined as an infinite recursive list. As Figure \ref{fig:taskI.1} (a) shows, the type of hamming function is a list of Int. This implementation using the map function to calculate 2x, 3x and
  5x and then merge them together as Figure \ref{fig:taskI.1} (b) demonstrated how it was evaluated.

  \section{Task I.2}

  \begin{figure}[H]
    \centering
    \centerline{\includegraphics[scale=0.6]{Sheet}}
    \caption{Extended Evaluator}
    \label{fig:sheet}
  \end{figure}

  Figure \ref{fig:sheet} demonstrates how evalCell function was extended to support Sum and Avg expression. The process of this implementation is given two CellRef c1 and c2, find every cell in between. Next, lookup corresponding values in the given sheet s. Then put all the values found in sheet s in list l. Finally using foldr to calculate the sum of all values.

  The similar idea was used to implement the evaluation of Avg expression but
  a further step was taken to yield the average value.

  The most obvious weakness of this evaluator is that it will be stuck in an infinite loop when there are circular references in the given sheet. As the type
  signature indicates the evalCell function will return a Double for every
  given sheet s and an Exp, which means when the Exp is Ref, it will keep
  evaluating until a Double can be returned. If there is a circular reference,
  this evaluator will keep evaluating without return.

  One simple solution to this problem is to maintain a separate list to keep
  track of the evaluation of Ref Exp. This list works like a stack but
  disallowing duplicated elements. A cell will be pushed to the stack if it contains Ref Exp as well as the reference cell. Then continue evaluating that reference cell,
  if it returns, which means there is no circular reference. Hence those cells can be popped off the stack. If a cell that contains Ref Exp already exists in the stack that there must be a circular reference. Therefore, the evaluator can just return fail without continue evaluating.

  \section{Task I.3}

  In skew binary random access list, a number is represented by a sequence of digits starting from least significant bit to the most significant bit. Each digit is associated with weight predefined by the positional system.
  
  \begin{figure}[H]
    \centering
    \centerline{\includegraphics[scale=0.6]{SBRAL-Drop}}
    \caption{Drop Function}
    \label{fig:drop}
  \end{figure}

  Figure \ref{fig:drop} shows the implementation of drop function which drops the first n elements from the list. The drop function takes an integer and an RList as input and returns an RList without the first n elements as output. If the tree is empty which means nothing can be dropped, hence an empty list is returned. If the input integer is zero which means nothing needs to be dropped, then the same list taken as input will be returned. If the size of the first element in the list equals the number of elements needs to be dropped, then the tail of the input list will be returned. If the number of
  elements need to be dropped is greater than the size of the first element in the
  list, then the first element will be dropped and the drop function will be
  called recursively to continue processing the remaining list. The operations
  mentioned above would run in constant time.

  The next case of the drop function determines the overall time complexity. If the number of elements needs to be dropped is less than the size of the first element in the list, then a helper function drop' will be called. The drop' function takes two integers and a tree as input and returns an RList. Similar to the first two cases of drop function, the first three cases of drop' function are practically doing nothing. Since the input tree is complete binary leaf tree, the size of the tree will be halved for every element dropped from the tree. Therefore, the overall time complexity is $O
  (log{}n)$. If the number of elements needs to be dropped is less than or equals to the half size of the input tree, then only the left children of the input tree will be processed. Otherwise, the left children will be dropped and the right children will be processed accordingly. The drop' function will be called recursively until the drop is finished or there is nothing to drop and return the final result.

  A test function called testDrop is created to test the drop function. The key idea is the RList which represents n should equal to the RList that represents
  2n without the first n elements.

  \section{Task I.4}
  
  \begin{figure}[H]
    \centering
    \centerline{\includegraphics[scale=0.4]{Intervals}}
    \caption{Intervals}
    \label{fig:intervals}
  \end{figure}
  
  Figure \ref{fig:intervals} shows how Ivl was made an instance of Num and
  Fractional class. As an instance of Num class, (+), (-) and abs functions were implemented according to the rule of interval arithmetic.

  As for (*) function, there are a few cases that NaN may be yield which means the arithmetic operation is illegal. Hence, the (*) function will return error every time NaN occurs. The signum function takes an Ivl as input and returns an Ivl with the signum value of lower bound and signum value of the upper bound. The
  fromIntegral function takes an int as input and returns an Ivl with the same
  lowwer and upper bound value converted from the input.

  The fromRational function in Fractional class works similarly as the
  fromIntegral function. The (/) function also employed the same manner to deal with NaN. Furthermore, in this implementation, divided by zero is not allowed.
  Hence, if divided by zero ever occurs, an error will be returned.

  As for the (+/-) function, the key idea is to calculate the symmetric interval around a specific number. The input numbers may be negative, to deal with this situation, the lower bound of the resulting Ivl is the smaller value of the result of plus and minus of the two input value while the upper bound is the larger value of the result of plus and minus of the two input values. A few
  test cases are provided in the testIvl function to test the result of the (+/-)
  function. 

  \section{Task I.5}

  \begin{figure}[H]

    \begin{minipage}[b]{0.48\linewidth}
      \centering
      \centerline{\includegraphics[width=8.0cm]{Stats}}
      % \vspace{1.5cm}
      \centerline{ (a) Recursive Statistics}\medskip
    \end{minipage}
    \hfill
    \begin{minipage}[b]{0.48\linewidth}
      \centering
      \centerline{\includegraphics[width=8.0cm]{StatsFoldMap}}
      % \vspace{1.5cm}
      \centerline{ (b) Statistics using foldMap}\medskip
    \end{minipage}
    % 
    \caption{TaskI.5 1 2}
    \label{fig:taskI.5.1.2}
    % 
  \end{figure}

  Figure \ref{fig:taskI.5.1.2} shows how the statistics functions were implemented using two different approaches. Figure \ref{fig:taskI.5.1.2} (a)
  illustrates how the statistics function was defined using recursion. The statistics function takes a Drawing Object as input and returns Statistics as a result by calling the accumStats function to do the computation. The
  accumStats was implemented using recursion by taking the result of last computation and the remaining Drawing Object as input and eventually return the final result. The area and circum functions were defined to aid the computation. They calculated the area and circumference of a single object repsectively.

  Figure \ref{fig:taskI.5.1.2} (b) shows a different approach of defining the
  statistics function by making Drawing an instance of Foldable and making
  AccumStats an instance of Semigroup. By making Drawing an instance of
  Foldable, a function can be applied to every object in drawing and the results can be combined. By making AccumStats an instance of Semigroup, it provided a
  way of merging two AccumStats to form a new AccumStats.

  As for the statistics using foldMap, the overall structure is similar to the
  recursive version, but the accumStats was defined differently. Instead
  processing all objects in Drawing, the accumStats now only need to convert one
  object to AccumStats and the results for all object in Drawing can be combined
  using foldMap.

  \begin{figure}[H]
    \centering
    \centerline{\includegraphics[scale=0.4]{StatsBonusBound}}
    \caption{newtype Bounded}
    \label{fig:bounded}
  \end{figure}

  Figure \ref{fig:bounded} shows the implementation details of Bounded class for
  Length and Area. They both bounded below by 0 and above by +Infinity.
  
  \begin{figure}[H]

    \begin{minipage}[b]{0.48\linewidth}
      \centering
      \centerline{\includegraphics[width=8.0cm]{StatsBonus}}
      % \vspace{1.5cm}
      \centerline{ (a) Recursive Statistics for newtype}\medskip
    \end{minipage}
    \hfill
    \begin{minipage}[b]{0.48\linewidth}
      \centering
      \centerline{\includegraphics[width=8.0cm]{StatsBonusFoldMap}}
      % \vspace{1.5cm}
      \centerline{ (b) Statistics using foldMap for newtype}\medskip
    \end{minipage}
    % 
    \caption{TaskI.5 3}
    \label{fig:taskI.5.3}
    % 
  \end{figure}

  Figure \ref{fig:taskI.5.3} demonstrates the two versions of the statistics function implemented according to Length and Area types. They work the same as
  the version shown in figure \ref{fig:taskI.5.1.2} but there are some
  workarounds according to the type definitions. For example, some values may be
  wrapped in Area then they need to be extracted and put back to Area after some
  computations. The same processes work for values in Length, Sum and Max.
  


\end{document}


