 \documentclass{article}
\usepackage{cite}
\usepackage{inputenc}
\usepackage{setspace}
\usepackage[margin=0.75in]{geometry}
\usepackage[style=numeric]{biblatex}
\addbibresource{../bibs/ref.bib}
\usepackage{float}
\usepackage{graphicx}
\graphicspath{ {./images/} }


\onehalfspace
\setlength{\parindent}{0pt}
\setlength{\parskip}{1em}



\begin{document}

\begin{center}
  \LARGE{\textbf{Real-world Functional Programming}} \\
  \Large{Coursework Part I Report} \\
  \normalsize{14274056 Junsong Yang (psyjy3)} \\
  \today
\end{center}


\begin{normalsize}
  \section{TaskI.1}
  The key ideas for this task are to explore the infinite data structure
  in haskell and to gain a better understanding of lazy evaluation in haskell.

  \section{TaskI.2}

  \section{TaskI.3}

  \section{TaskI.4}

  \section{TaskI.5}



\end{document}


