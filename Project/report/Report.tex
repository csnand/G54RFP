\documentclass{article}
\usepackage{inputenc}
\usepackage{setspace}
\usepackage[margin=0.75in]{geometry}
\usepackage[style=numeric]{biblatex}
\addbibresource{ref.bib}
\usepackage{float}
\usepackage{graphicx}
\graphicspath{ {./images/} }


\onehalfspace
\setlength{\parindent}{0pt}
\setlength{\parskip}{1em}



\begin{document}

\begin{center}
  \LARGE{\textbf{Real-world Functional Programming}} \\
  \Large{Project Report} \\
  \normalsize{14274056 Junsong Yang (psyjy3)} \\
  \today
\end{center}


\begin{normalsize}
  \section{Introduction}
  In this section, the general idea about this project will be introduced as
  well as how this project would fit for this module.

  The project is called recipe house, which is a recipe recommendation service
  based on the ingredients.The logic behind it is quite simple. It would require
  users to input whatever ingredients they have, then it will return recipes
  that used those ingredients. This project includes two parts: a back-end
  server and a web interface. Since I don't have any recipe database, the
  back-end server will query a third-party API for recipe info and trim the
  returned data in JSON and return the trimmed data as a response to the web
  interface.

  The third party APIs that I will use are provided by Edamam. They have over 2
  millions of recipes specified by diets, calories, nutrient ranges and simply
  just ingredients.

  This project idea was used to compete in ATOS IT Challenge and was
  shortlisted(top 20 worldwide). The back-end was written in Go as well as the
  web interface with a bit vanilla javascript, but it was not quite finished.
  Hence, this idea will be reimplemented in a different approach for this
  project.

  As for how this project may fit for 10 credits:
  \begin{itemize}
  \item[]{learning scala for backend (about 20 hrs)}
  \item[]{learning React (with any host language) (20 hrs)}
  \item[]{revisit web technologies (HTML CSS) (10 hrs)}
  \item[]{implementation of the back-end (15 hrs)}
  \item[]{implementation of the web interface (15 hrs)}
  \item[]{report writing (20 hrs)}  
  \end{itemize}

  A successful project would be a working web interface that allowed user to use
  either text-based or image based search for recipes based on what ingredients
  they have. The front-end would sent the ingredients info to the back-end
  server via REST API. Then the back-end server will query the third-party API
  for recipes and parse the returned data and sent back to the front-end as
  response. The front-end will present the response from server to the end user. 


  \section{Technical Background}
  In this section, the technological choices made for both the back end and
  front end will be discussed in detail with justification.

  For backend server, the language of choice is Scala. Scala is a multi-paradigm
  programming language compiles to java bytecode and supports both functional
  programming and imperative programming. Hence, by using scala, we can not only
  explore functional programming in depth but also leveraging the java
  ecosystem.

  As for the frontend interface, a javascript library called React will be used.
  The reason for that choice is that, React supports functional reactive
  programming paradigm which can be used to create graphical user interface.
  

  \section{Implementation}
  In this section, the general architecture of this project will be introduced
  as well as some essential implementation details.

  
  \section{Reflection}
  
\end{document}
